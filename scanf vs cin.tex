\documentclass[11pt, a4paper]{article} % A4 paper size by default, use 'letterpaper' for US letter
\usepackage{geometry}

\geometry{left=2cm, top=1.4cm, right=2cm, bottom=2cm} % Configure page margins with geometry
\usepackage[utf8]{inputenc}

\usepackage{helvet}
\usepackage{listings}
\usepackage{xcolor}

\lstset{language=C++,
                basicstyle=\ttfamily,
                keywordstyle=\color{blue}\ttfamily,
                stringstyle=\color{red}\ttfamily,
                commentstyle=\color{green}\ttfamily,
                morecomment=[l][\color{magenta}]{\#}
}

\begin{document}
iostream version:
\begin{lstlisting}
#include <iostream>

int main(int argc, char **argv) {

  int parity = 0;
  int x;

  while (std::cin >> x)
    parity ^= x;
  std::cout << parity << std::endl;

  return 0;
}
\end{lstlisting}
scanf version
\begin{lstlisting}
#include <stdio.h>

int main(int argc, char **argv) {

  int parity = 0;
  int x;

  while (1 == scanf("%d", &x))
    parity ^= x;
  printf("%d\n", parity);

  return 0;
}
\end{lstlisting}
Result:
Using a third program, I generated a text file containing 33,280,276 random numbers. The execution times are:
\begin{lstlisting}
iostream version:  24.3 seconds
scanf version:      6.4 seconds
\end{lstlisting}
The speed difference is largely due to the iostream I/O functions maintaining synchronization with the C I/O functions. We can turn this off with a call to 
\begin{lstlisting}
std::ios::sync_with_stdio(false);
\end{lstlisting}

\begin{lstlisting}
#include <iostream>

int main(int argc, char **argv) {

  int parity = 0;
  int x;

  std::ios::sync_with_stdio(false);

  while (std::cin >> x)
    parity ^= x;
  std::cout << parity << std::endl;

  return 0;
}
\end{lstlisting}
New result:
\begin{lstlisting}
iostream version:                       21.9 seconds
scanf version:                           6.8 seconds
iostream with sync_with_stdio(false):    5.5 seconds
\end{lstlisting}
C++ iostream wins! It turns out that this internal syncing / flushing is what normally slows down iostream i/o. If we're not mixing cstdio and iostream, we can turn it off, and then iostream is fastest.
\end{document}